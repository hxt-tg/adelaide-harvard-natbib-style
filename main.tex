\documentclass[a4paper]{article}

% Margins
\usepackage[a4paper,left=3cm,right=3cm,top=2.5cm,bottom=2.5cm]{geometry}

% AMS Packages
\usepackage{amsmath,amssymb,amsfonts,amsbsy}

% Links
\usepackage[hyphens]{url}
\usepackage{hyperref}
\hypersetup{colorlinks=true,allcolors=blue}

% References
\usepackage{natbib}
% \providecommand\adelaideyearleft{(}
% \providecommand\adelaideyearright[1]{)}
\setcitestyle{aysep={}}
\bibliographystyle{uniadelaide}

% Random Texts (Can be removed)
\usepackage{lipsum}

% Author info
\title{uniadelaide.bst - natbib style for UniAdelaide}
\author{hxt-tg}
\date{March 13, 2025}


\begin{document}
\maketitle

\begin{abstract}
    Your abstract goes here...

    ...
\end{abstract}

{\hypersetup{hidelinks} \tableofcontents} % This may not be neccessary?

\section{Structure}
\label{sec:structure}  % Make a label which lets this section can be "\ref"ed in the article.

\lipsum[1-2]

\subsection{Top Matter}

\lipsum[3-4]

\subsubsection{Article Information}

\lipsum[5-6]

\section{Citation test}

You should always use \verb|\citep{}| to produce citations.

This page gives a brief introduction on LaTeX \citep{wikibooksLaTeXDocStruct} and on BibTeX \citep{overleafBibTeX}.

This is a cite of UniAustralia Harvard style \citep[p.~2]{vaswani2017attention} (cite for page 2) and \citep[pp.~3-4]{vaswani2017attention} (cite for multiple pages 3$\sim$4). Or without page \citep{vaswani2017attention}.

\begin{enumerate}
    \item This is a citation for one author \citep{ticker2017music}.
    \item Two authors \citep{habel2009academic}.
    \item Three or more \citep{mnih2015human}.
    \item You can use this website to generate bibtex from URL \citep{getbibtexBibTeXGenerator}.
\end{enumerate}

And This is a ref to the first section \ref{sec:structure}. 

The blue hyperlink on the correct reference need \emph{TWICE BUILD} the .tex file. (In VSCode this means click build twice.) Because the FIRST BUILD produces the contents and reference position information. The SECOND BUILD makes links.

\bibliography{biblio}

\end{document}